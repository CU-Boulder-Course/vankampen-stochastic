\leveldown{Cumulative pdf - pg. 4}

\leveldown{Problem}

Show that $\mathbb{P}(x)$ must be a monotone non-decreasing function with $\mathbb{P}(-\infty) = 0$ and $\mathbb{P}(+\infty) = 1$.
What is its relation to $\mathcal{P}$?

\levelstay{Solution}

That $\mathbb{P}$ is monotone non-decreasing is obvious from the fact that $P$ is positive, because
\begin{equation*}
\mathbb{P}(x) = \int_{-\infty}^x \, P(x') \, dx' \, .
\end{equation*}
It's also obvious, for the same reason (and from the normalization condition), that $\mathbb{P}(-\infty)=0$ and $\mathbb{P}(+\infty) = 1$.

The relation between $\mathbb{P}$ and $\mathcal{P}$ is
\begin{equation*}
\mathbb{P}(x) = \mathcal{P}((-\infty, x)) \, .
\end{equation*}
