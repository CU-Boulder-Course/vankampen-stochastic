\leveldown{Dice which sum to 9 - pg. 11}

\leveldown{Problem}

Two dice are thrown and the outcome is 9.
What is the probability distribution of the first die conditional on the given total?
Why is this result not incompatible with the obvious fact that the two dice are independent?

\levelstay{Solution}
The ways to get a total of 9 are:
\begin{equation*}
(3, 6), \, (4, 5), \, (5, 4), \, (6, 3) \, .
\end{equation*}
Therefore, the probability distribution $P(n)$ for the first die to show $n$ points is
\begin{equation*}
P(1)=P(2)=0, \, P(3)=P(4)=P(5)=P(6)=1/4 \, .
\end{equation*}
This is not incompatible with the fact that the dice are independent because the fact that the total is 9 is a constraint which does not exist when describing a-priori probabilities.
