\leveldown{Opinion poll - pg. 4}

\leveldown{Problem}

An opinion poll is conducted in a country with many political parties.
How large a sample is needed to be reasonably sure that a party of 5 percent will show up in it with a percentage between 4.5 and 5.5?

\levelstay{Solution}

This comes down to integrating the probability density of a random walk process with a large number of trials and a fixed 5\% probability of success on each trial.

Consider a string of $N$ polls.
Each one is either ``success'', meaning the person polled is from the party, or ``fail'', meaning that the person is not from the party.
The probability of success is $p$ and the probability of failure is $q = 1 - p$.
The probability of getting $n$ successes is
\begin{equation*}
P(n) = p^n q^{N-n} \frac{N!}{n! (N-n)!} \, .
\end{equation*}
It's a reasonably well known result that if $N$ is big enough, $P(n)$ is well approximated as\footnote{Use Stirling's approximation to prove this.}
\begin{equation*}
P(n) \approx \frac{1}{\sqrt{2\pi \sigma^2}} \exp \left[ - \frac{(N - np)^2}{2 \sigma^2}\right]
\end{equation*}
where $\sigma^2 = Npq$.
The probability that the fraction of successes is between 4.5\% and 5.5\% is
\begin{equation*}
K \equiv \text{Probability}(0.045\,N < n < 0.055\,N) = \sum_{n=0.045 \cdot N}^{0.055 \cdot N} P(n) \, .
\end{equation*}
I don't think we can do this analytically (although there may be some tricky way to approximate the result).
Instead, we can evaluate the sum numerically for a few values of $N$ and see where we get, say, 90\% probability.
Doing this, we find that we get 90\% probability at around $N=5,200$.
A python script \texttt{opinion\_poll.py} for the numerical analysis is included with the source repo.

Note that it's helpful to approximate the distribution as a Gaussian to make the numerics simple.
Otherwise, we'd have to compute really large factorials.

It would be very nice to have a real analytic treatment of this problem so that we know the functional dependence on $N$.
From the numerics you can see that the dependence of $K$ on $N$ seems to follow the form $1 - \exp(-N/N_0)$ for some $N_0$.
